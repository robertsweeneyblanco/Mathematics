\documentclass{beamer}
\usetheme{Berkeley}
 
\usepackage[utf8]{inputenc}
\usepackage{amssymb}
\usepackage{amsmath} %piecewise function%
 
 
%Information to be included in the title page:
\title{Modular Forms}
\author{Robert Sweeney Blanco}
\institute{Advisor: Brandon Williams}
\date{December 5, 2017}
 
 
 
\begin{document}
 
\frame{\titlepage}
 
\begin{frame}
\frametitle{Defining Modular Forms}
In this presentation, I will be working with the modular group $SL_2(\mathbb{Z}) = \Gamma_1$
\begin{block}{Definition: Modular form}
$f$ is a modular form of weight $k$ if 
\begin{enumerate}
\item  f is holomorphic on $\mathbb{H}$\\
\item f continues to be holomorphic as $z \longrightarrow i\infty$
\item $f(\frac{az+b}{cz+d}) = (cz+d)^k f(z) \  \ \ \forall \begin{bmatrix}
a & b \\
c & d \\
\end{bmatrix} 
\in \Gamma_1$
\end{enumerate}
\end{block}
\end{frame}

\begin{frame}
\frametitle{Simplifying the full Modular Group}
\begin{block}{Theorem}
$\Gamma_1$ is a group generated by $S = \begin{bmatrix}
0 & 1 \\
-1 & 0 \\
\end{bmatrix}$
and $T = \begin{bmatrix}
1 & 1 \\
0 & 1 \\
\end{bmatrix}$
\end{block}
This simplifies proving a function is a modular form when working with the full modular group because now we only need to check that the desired behavior for composition with Mobius transformations from the matrix group works for the transformations corresponding with $S$ and $T$. That is $S(z)=f(\frac{-1}{z}) = z^kf(z)$ and $T(z)=f(z+1)=f(z)$. Note this means modular forms are periodic and have Fourier series!
\end{frame}

\begin{frame}

\frametitle{Dimension of $M_k(\Gamma_1)$}
Modular forms of a fixed weight form a vector space over the Complex numbers. The dimension of $M_k(\Gamma_1)$ is $0$ for all negative and odd values of $k$. Otherwise the dimension is given by the following formula:
\[   
dim(M_k(\Gamma_1)) = 
     \begin{cases}
       \lfloor \frac{k}{12} \rfloor + 1 & \quad \text{if} \  k \not \equiv  12 \ mod(12) \\
       \lfloor \frac{k}{12} \rfloor & \quad \text{if} \  k \equiv  12 \ mod(12) \\
     \end{cases}
\]
\end{frame}

%Eisenstein Series%
\begin{frame}

\frametitle{Defining Eisenstein Series}
There are different ways of defining the Eisenstein Series, but they all differ by a constant. Here is a common definition for series of even weight $k>2$: 
\begin{block}{Eisenstein Series}
$$G_k(\tau) = \sum_{\substack{(m,n) \in \mathbb{Z}^2 \\ (m,n) \neq (0,0)}}  \frac{1}{(m \tau + n)^k}$$
\end{block}
This series converges absolutely for $k>2$.

\end{frame}

%Proving Eisenstein Series transforms under T%
\begin{frame}
\frametitle{Transformation under $T$}

$$G_k(\tau + 1) = \sum_{(m,n) \in \mathbb{Z}^2/(0,0)}  \frac{1}{(m (\tau +1) + n)^k}$$\\
$$ = \sum_{(m,n) \in \mathbb{Z}^2/(0,0)}  \frac{1}{(m \tau + m +  n)^k}$$ \\
Define $\mu=m+n$ (Note $(m,n) \neq (0,0) \Rightarrow (m, \mu) \neq (0,0)$) \\
$$=  \sum_{(m,\mu) \in \mathbb{Z}^2/(0,0)}  \frac{1}{(m \tau + \mu)^k} = G_k(\tau)$$



\end{frame}

%Proving Eisenstein Series are modular forms%
\begin{frame}

\begin{enumerate}
\item Since the series converges uniformly on any compact subset of $\mathbb{H}$, it is holomorphic
\item The series transforms properly for $S$ and $T$ $\Rightarrow$ satisfies Modularity Condition


 
\end{enumerate}

\end{frame}

\end{document}
